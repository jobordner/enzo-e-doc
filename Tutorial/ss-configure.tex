%======================================================================
\NEWSEC
%======================================================================

\subsection{\ssConfigure}


%    SConstruct
%    config/linux_blah

%    CELLO_PREC
%    CELLO_ARCH
%    (move prec to config?)
%   use online user documentation
%----------------------------------------------------------------------

\begin{frame}[fragile,label=ss-configure] 
\secframetitle{\ssConfigure}
%\framesubtitle{Configuring Enzo-P/Cello: \code{config/*.py}}
\footnotesize
\begin{enumerate}
\item Use or create a small config/*.py machine file
\item Set CELLO environment variables
\item Edit SConstruct user-configuration section (optional)
\end{enumerate}
\begin{semiverbatim}
 \uncover<1->{\prompt \redcode{cd cello-src}}\cursor{1}  
 \uncover<2->{\prompt \redcode{ls config}}\cursor{2} 
 \uncover<3->{\greencode{davros_gnu_debug.py   gordon_gnu.py    linux_gprof.py   ncsa_bw.py}}
 \uncover<3->{\greencode{davros_gnu.py         gordon_intel.py  linux_mpe.py}}
 \uncover<3->{\greencode{faraday_gnu_debug.py  gordon_pgi.py    mf_gnu_debug.py}}
 \uncover<3->{\greencode{faraday_gnu.py        linux_gnu.py     mf_gnu.py}}
 \uncover<4->{\prompt \redcode{export CELLO_ARCH=linux_gnu}}\cursor{4}  
 \uncover<5->{\prompt \redcode{export CELLO_PREC=single}}\cursor{5}  
 \uncover<6->{\prompt \redcode{cat config/linux_gnu.py}}\cursor{6}  
\end{semiverbatim}
\end{frame}

%----------------------------------------------------------------------
\begin{frame}[fragile] 
\secframetitle{\ssConfigure}
\framesubtitle{Configuring Enzo-P/Cello: \code{config/*.py}}
\tiny
\begin{semiverbatim}
 \uncover<1->{\prompt \redcode{cat config/linux_gnu.py}}\cursor{1}  
 \uncover<2->{import os}
 \uncover<2->{}
 \uncover<2->{is_arch_valid = 1}
 \uncover<2->{}
 \uncover<2->{flags_arch = '-Wall -O3 -g'}
 \uncover<2->{flags_link_charm = ' -rdynamic' # required for backtraces}
 \uncover<2->{}
 \uncover<2->{cc  = 'gcc '}
 \uncover<2->{f90 = 'gfortran'}
 \uncover<2->{}
 \uncover<2->{flags_prec_single = ''}
 \uncover<2->{flags_prec_double = '-fdefault-real-8 -fdefault-double-8'}
 \uncover<2->{}
 \uncover<2->{libpath_fortran = '.'}
 \uncover<2->{libs_fortran    = ['gfortran']}
 \uncover<2->{}
 \uncover<2->{use_papi=1}
 \uncover<2->{papi_inc = '/usr/local/include'}
 \uncover<2->{papi_lib = '/usr/local/lib'}
 \uncover<2->{hdf5_inc     = '/usr/include'}
 \uncover<2->{hdf5_lib     = '/usr/lib'}
 \uncover<2->{png_path     = '/lib/x86_64-linux-gnu'}
 \uncover<2->{}
 \uncover<2->{home = os.environ['HOME']}
 \uncover<2->{charm_path   = home + '/Charm/charm'}
 \uncover<2->{grackle_path = home + '/Software/Grackle/src/clib'}
\end{semiverbatim}
\end{frame}

%----------------------------------------------------------------------
\begin{frame}[fragile] 
\secframetitle{\ssConfigure}
\framesubtitle{Configuring Enzo-P/Cello: \code{SConstruct}}
\tiny
\begin{semiverbatim}
\uncover<1->{\prompt \redcode{gedit SConstruct}}\cursor{1}  

#======================================================================
# USER CONFIGURATION
#======================================================================

#----------------------------------------------------------------------
# Whether to print out detailed messages with the TRACE() series of statements
#----------------------------------------------------------------------

trace = 0

#----------------------------------------------------------------------
# Whether to trace main phases
#----------------------------------------------------------------------

verbose = 0

#----------------------------------------------------------------------
# Whether to print out messages with the TRACE_CHARM() and TRACEPUP()
#  series of statements
#----------------------------------------------------------------------

trace_charm = 0

\end{semiverbatim}
\end{frame}

%----------------------------------------------------------------------
\begin{frame}[fragile] 
\secframetitle{\ssConfigure}
\framesubtitle{Configuring Enzo-P/Cello: \code{SConstruct}}
\tiny
\begin{semiverbatim}
\uncover<1->{\prompt \redcode{gedit SConstruct}}\cursor{1}  

#----------------------------------------------------------------------
# Whether to enable displaying messages with the DEBUG() series of
# statements. Also writes messages to out.debug.<P> where P is the
# (physical) process rank. Still requires the "DEBUG" group to be
# enabled in Monitor (that is Monitor::is_active("DEBUG") must be true
# for any output)
#----------------------------------------------------------------------

debug = 0

#----------------------------------------------------------------------
# Whether to periodically print all field values.  See
# src/Field/field_FieldBlock.cpp
#----------------------------------------------------------------------

debug_verbose = 0

\end{semiverbatim}
\end{frame}

%----------------------------------------------------------------------
\begin{frame}[fragile] 
\secframetitle{\ssConfigure}
\framesubtitle{Configuring Enzo-P/Cello: \code{SConstruct}}
\tiny
\begin{semiverbatim}
\uncover<1->{\prompt \redcode{gedit SConstruct}}\cursor{1}  

#----------------------------------------------------------------------
# Whether to track dynamic memory statistics.  Can be useful, but can
# cause problems on some systems that also override new [] () / delete
# [] ()
#----------------------------------------------------------------------

memory = 1

#----------------------------------------------------------------------
# Enable charm++ dynamic load balancing
#----------------------------------------------------------------------

balance = 1

\end{semiverbatim}
\end{frame}

%----------------------------------------------------------------------
\begin{frame}[fragile] 
\secframetitle{\ssConfigure}
\framesubtitle{Configuring Enzo-P/Cello: \code{SConstruct}}
\tiny
\begin{semiverbatim}
\uncover<1->{\prompt \redcode{gedit SConstruct}}\cursor{1}  
#----------------------------------------------------------------------
# Whether to compile with -pg to use gprof for performance profiling
#----------------------------------------------------------------------

use_gprof = 0

#----------------------------------------------------------------------
# Whether to compile with the Grackle chemistry and cooling library
#
# WARNING: must update grackle-related lines in src/Enzo/enzo.ci
#----------------------------------------------------------------------

use_grackle = 0

#----------------------------------------------------------------------
# Whether to run the test programs using valgrind to check for memory leaks
#----------------------------------------------------------------------

use_valgrind = 0

\end{semiverbatim}
\end{frame}

%----------------------------------------------------------------------
\begin{frame}[fragile] 
\secframetitle{\ssConfigure}
\framesubtitle{Configuring Enzo-P/Cello: \code{SConstruct}}
\tiny
\begin{semiverbatim}
\uncover<1->{\prompt \redcode{gedit SConstruct}}\cursor{1}  
#----------------------------------------------------------------------
# Whether to use Cello Performance class for collecting performance
# data (currently requires global reductions, and may not be fully
# functional) (basic time data on root processor is still output)
#----------------------------------------------------------------------

use_performance = 0

#----------------------------------------------------------------------
# Whether to compile the CHARM++ version for use with the Projections
# performance tool.
#----------------------------------------------------------------------

use_projections = 0

#----------------------------------------------------------------------
# How many processors to run parallel unit tests
#----------------------------------------------------------------------

ip_charm = '4'

#----------------------------------------------------------------------
# Whether this is a Mercurial repository
#----------------------------------------------------------------------

have_mercurial = 1
\end{semiverbatim}
\end{frame}

