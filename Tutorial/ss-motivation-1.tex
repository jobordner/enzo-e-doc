\begin{frame}[fragile,label=ss-motivation] 
\secframetitle{\ssMotivation}
% One might ask why am I working on Enzo-P and Cello.  Enzo is clearly
% a very powerful code, capable of being applied to a wide range
% of problems in astrophysics and cosmology.

% [PHYSICS] This is due in part to Enzo's wide range of physics
% capabilites, from hydrodynamics and self-gravity, to cosmological
% expansion, chemistry and cooling, star formation, MHD, RHD, and so
% on.

% [METHODS] Enzo contains an even wider range of numerical methods,
% which serve to discretize and solve these equations.  Most of the
% physics capabilities in Enzo can be solved using one of multiple
% available methods, such as PPM, ZEUS or MUSCL hydrodynamics,
% Gadget, Cloudy, and Grackle chemistry and cooling, and implicit
% coupled FLD and Enzo+Moray RHD

% [DATA] All of Enzo's methods are in turn built on its parallel
% adaptive mesh refinement data structure.  Patches in the AMR
% hierarchy contain both particle data and mesh data, so both
% Lagrangian and Eularian methods are supported, as well as hybrid
% particle-mesh methods.

% [SLIDE] Together, these three layers enable Enzo to do numerical
% astrophysics.  The equations serve to mathematically approximate
% physical phenomena, the numerical methods serve to define how to
% represent required data on a computer and proceedures for finding
% numerical solutions to the equations, and the data structures serve
% to represent the actual numerical solution including all
% intermediate computations within the parallel computer.
  \framesubtitle{\enzo's strengths}
\centerline{\textbf{\only<1>{Applicable to a wide range of astrophysical/cosmological problems}
\only<2>{Implements a variety of sophisticated numerical methods}
\only<3>{Enabled by adaptive mesh refinement with particles and fields}}}
\setbeamercolor{block title}{bg=red!30,fg=black}
\begin{block}<+->{\textbf{Physics Equations}: \textit{mathematical models}}
   \textcolor{red!80!black}{
 \footnotesize
    \textbullet\ Hydrodynamics (Euler equations)
    \textbullet\ Gravity ($\nabla^2\Phi=4\pi G\rho$)
    \textbullet\ Chemistry/cooling
    \textbullet\ Star formation
    \textbullet\ Magnetism
    \textbullet\ Radiation \Large $\ldots$
    }
\end{block}

\setbeamercolor{block title}{bg=green!30,fg=black}
\begin{block}<+->{\textbf{Numerical Methods}: \textit{approximate and solve}}
    \textcolor{green!50!black}{
\footnotesize     \textbullet\ PPM, \textbullet\ ZEUS, \textbullet\ MUSCL
     \textbullet\ FFT \textbullet\ multigrid
     \textbullet\ Gadget cooling
     \textbullet\ Cloudy cooling
     \textbullet\ Grackle
     \textbullet\ Dedner MHD
     \textbullet\ MHD-CT
     \textbullet\ Implicit FLD
     \textbullet\ Moray \Large $\ldots$
     }
  \end{block}

\setbeamercolor{block title}{bg=blue!30,fg=black}
\begin{block}<+->{\textbf{Data Structures}: \textit{computer representation}}
    \textcolor{blue}{
\footnotesize
    \textbullet\ Structured Adaptive Mesh Refinement (SAMR)
    \textbullet\ Eularian fields
    \textbullet\ Lagrangian particles
 \Large $\ldots$}
  
\end{block}

\end{frame}
